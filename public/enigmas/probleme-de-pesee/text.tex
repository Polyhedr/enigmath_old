%logique
\section*{Problème de pesée}
\begin{center}
    \includegraphics[width=0.4\textwidth]{\currfiledir/image.jpg}
\end{center}
\subsection*{Énoncé}
Vous disposez d'un certain nombre de boules identiques à vu d'oeil, et toutes de même poids, à l'exception d'une (l'intrus) qui a un poids légèrement différent. Le but est d'identifier l'intrus à l'aide d'une balance à plateaux qui indique simplement si deux groupes de boules sont de même poids ou si l'un est plus lourd que l'autre.

\medskip
\textbf{Questions} :
\begin{enumerate}
\item\indicators{1.2}{0} En sachant que l'intrus est plus légère, l'identifier parmis 27 boules en trois pesées maximum.
\item\indicators{2.7}{0} Cette fois vous ne savez pas si l'intrus est plus lourde ou plus légère, l'identifier parmis 13 boules en trois pesées maximum.
\item \indicators{3.4}{0} Généralisation : En notant $P$ le nombre de pesées autorisées, exprimer en fonction de $P$ le nombre maximal $N$ de boules tel qu'il existe une stratégie permettant de determiner l'intrus parmis $N$ boules en au plus $P$ pesées (on ne sait pas si l'intrus est plus léger ou plus lourd).
%\item \emoji{hot-pepper}\emoji{hot-pepper}\emoji{hot-pepper} Que se passe-t-il s'il y a au plus deux intrus identiques parmi $N=11$ boules, dont les poids diffèrent du poids standard d'une même valeur fixe inconnue (positive ou négative)~?
\end{enumerate}
\subsection*{Solution}
\begin{enumerate}
    \item \label{En sachant que l'intrus est plus légère}
On peut identifier une boule plus légère parmi \( 3^k \) boules avec \( k \) pesées (cette méthode fonctionne aussi pour une boule plus lourde).
Dans le cas de 27 boules, on procède ainsi :
\begin{itemize}
    \item On divise les 27 boules en 3 groupes de 9.
    \item On compare deux groupes sur la balance :
    \begin{itemize}
        \item s’ils sont égaux, la boule légère est dans le troisième groupe ;
        \item sinon, elle est dans le groupe le plus léger.
    \end{itemize}
    \item On répète ce procédé avec le groupe de 9 sélectionné (3 groupes de 3), puis enfin avec les 3 dernières boules.
\end{itemize}
Ainsi, en 3 pesées, on isole successivement un groupe de 9, puis de 3, puis la boule plus légère.
\item 
  \textbf{Première pesée} : On divise les 13 boules en trois groupes :
  \begin{itemize}
    \item Deux groupes de 4 boules chacun,
    \item Un groupe de 5 boules restantes.
  \end{itemize}
  On pèse les deux groupes de 4 boules l'un contre l'autre :
  \begin{itemize}
    \item Équilibre : la boule différente se trouve dans le groupe de 5. Passer à l'étape 2a.
    \item Déséquilibre : la boule différente se trouve dans les 8 boules pesées. Passer à l'étape 2b.
  \end{itemize}

  \textbf{Deuxième pesée (cas 2a)} : On sélectionne 3 boules du groupe de 5, que l’on compare à 3 boules quelconques (de masse connue) du groupe équilibré :
  \begin{itemize}
    \item Équilibre : la boule différente est parmi les 2 restantes. Pour la \textbf{troisième pesée}, on compare l’une des deux à une boule connue. Si équilibre, c’est la boule non testée ; sinon, c’est celle sur la balance.
    \item Déséquilibre : la boule différente est parmi les 3 testées, et le sens du déséquilibre indique si elle est plus légère ou plus lourde. Pour la \textbf{troisième pesée}, on applique alors la méthode décrite dans la question~\ref{En sachant que l'intrus est plus légère} pour 3 boules et une pesée ($k=1$).
  \end{itemize}

  \textbf{Deuxième pesée (cas 2b)} : On retire trois boules du plateau le plus lourd, on transfère trois boules du plateau le plus léger vers le plus lourd, et on place trois boules non encore pesées sur le plateau le plus léger. On mémorise bien la provenance de chaque boule. 
Trois cas peuvent alors se présenter~:
\begin{itemize} 
  \item Le même plateau reste plus lourd~: cela signifie soit que la boule restée sur ce plateau est plus lourde, soit que celle restée sur l’autre plateau est plus légère. Une \textbf{troisième pesée} entre une de ces deux boules et une des 11 autres permet de conclure. 
  
  \item Le plateau initialement plus lourd devient plus léger~: cela implique que l’une des trois boules transférées est plus légère. Pour la \textbf{troisième pesée}, on applique donc la question~\ref{En sachant que l'intrus est plus légère}($k=1$).
  
  \item Les deux plateaux s’équilibrent~: cela signifie que la boule différente est l’une des trois retirées du plateau initialement plus lourd, et qu’elle est plus lourde. Pour la \textbf{troisième pesée}, on applique donc la question~\ref{En sachant que l'intrus est plus légère} ($k=1$).
\end{itemize}
\item \textbf{Solution pour \(
3 \leq N \leq \frac{3^P - 1}{2}
\)} :  
On se place dans $\{0,1,2\}^P$ et on définit l’ensemble $A$ comme l’ensemble des 
\(
\frac{3^P - 3}{2}
\)
vecteurs de $\{0,1,2\}^P$ dont le premier changement de chiffre après le début est un changement de \( 0 \) à \( 1 \), de \( 1 \) à \( 2 \), ou de \( 2 \) à \( 0 \).
Par exemple, pour $P = 3$, $A$ contient
\[
\begin{bmatrix}
0 \\ 0 \\ 1
\end{bmatrix},
\begin{bmatrix}
0 \\ 1 \\ 0
\end{bmatrix},
\begin{bmatrix}
0 \\ 1 \\ 1
\end{bmatrix},
\begin{bmatrix}
0 \\ 1 \\ 2
\end{bmatrix},
\begin{bmatrix}
1 \\ 1 \\ 2
\end{bmatrix},
\begin{bmatrix}
1 \\ 2 \\ 1
\end{bmatrix},
\begin{bmatrix}
1 \\ 2 \\ 2
\end{bmatrix},
\begin{bmatrix}
1 \\ 2 \\ 0
\end{bmatrix},
\begin{bmatrix}
2 \\ 2 \\ 0
\end{bmatrix},
\begin{bmatrix}
2 \\ 0 \\ 2
\end{bmatrix},
\begin{bmatrix}
2 \\ 0 \\ 0
\end{bmatrix},
\begin{bmatrix}
2 \\ 0 \\ 1
\end{bmatrix}.
\]
Les vecteurs de $A$ peuvent être regroupés en
\(
\frac{3^{P-1} - 1}{2}
\)
groupes de trois, chaque groupe étant obtenu par permutation cyclique des chiffres $0,1,2$ (c’est-à-dire en remplaçant simultanément les $0$ en $1$, les $1$ en $2$, et les $2$ en $0$).  
Par exemple, pour $P = 3$, l’ensemble $$\bigg\{
\begin{bmatrix} 0 \\ 0 \\ 1 \end{bmatrix},
\begin{bmatrix} 1 \\ 1 \\ 2 \end{bmatrix},
\begin{bmatrix} 2 \\ 2 \\ 0 \end{bmatrix}
\bigg\}$$ forme un tel groupe.

On étiquette une boule par le vecteur ne contenant que des~$1$, 
et les $N-1$ boules restantes par les vecteurs de $A$, en respectant la règle suivante : chaque groupe de trois est utilisé entièrement avant de passer au suivant.  
À la fin, il peut rester une ou deux boules issues d’un groupe incomplet : s’il reste une boule, on choisit celle dont le premier chiffre est $1$ ; s’il en reste deux, on prend celles dont les premiers chiffres sont $0$ et $2$.

Pour $1 \leq i \leq P$ et $d \in \{0,1,2\}$, on note $C(i,d)$ la classe des boules dont l’étiquette a le chiffre~$d$ en $i$-ème position.  
Par construction, $C(1,0)$ et $C(1,2)$ ont le même cardinal.

Les $P$ pesées sont définies comme suit : lors de la $i$-ème pesée, les boules de $C(i,0)$ sont placées sur le plateau gauche, celles de $C(i,2)$ sur le plateau droit, et celles de $C(i,1)$ sont mises de côté.  
À l’issue de la première pesée, on sait qu’un des ensembles $C(1,0)$, $C(1,1)$ ou $C(1,2)$ ne contient pas l’intrus.  
Pour $i > 1$, il peut manquer au plus une boule dans $C(i,0)$ ou $C(i,2)$ ; ce manque est compensé en ajoutant sur un plateau une boule déjà identifiée comme non-intrus lors de la première pesée (les ensembles $C(1,0)$, $C(1,1)$ et $C(1,2)$ sont non vides, car $N \geq 3$).

On code le résultat de la $i$-ème pesée par un chiffre $a_i \in \{0,1,2\}$ :  
\[
a_i \triangleq
\begin{cases}
0 & \text{si le plateau gauche descend},\\
1 & \text{si la balance reste équilibrée},\\
2 & \text{si le plateau droit descend}.
\end{cases}
\]
On obtient ainsi un vecteur $a \triangleq (a_1,\dots,a_P)$.  
Du résultat de la $i$-ème pesée, on déduit que l’intrus est soit plus lourde et a $a_i$ comme $i$-ème chiffre de son étiquette, soit plus légère et a $2 - a_i$ comme $i$-ème chiffre de son étiquette.

Après $P$ pesées, l’intrus est donc identifiée de manière unique comme étant la boule d’étiquette~$a$ ou $2-a$.

\textbf{Insolubilité pour $N>\frac{3^P-1}{2}$} : Supposons qu'une solution soit possible pour $N$ boules.  
Lors de la $i$-ème pesée, on note
\(
C(i,0) \) l'ensemble des boules placées dans le plateau gauche, \(
C(i,2)\) l'ensemble des boules placées dans le plateau droit, et \(
C(i,1)\) l'ensemble des boules non placées sur la balance.
La solution doit fournir une méthode de choix des ensembles $C(i,0)$, $C(i,1)$ et $C(i,2)$, en ne se basant que sur les résultats des $(i-1)$ premières pesées.
On définit :
\[
a_i \triangleq
\begin{cases}
0 & \text{si le plateau gauche descend},\\
1 & \text{si la balance reste équilibrée},\\
2 & \text{si le plateau droit descend},
\end{cases}
\]
pour la $i$-ème pesée. La somme des connaissances acquises après les $i$ premières pesées est que l'intrus est soit plus lourde et appartient à $\cap_{j=1}^i C(j,a_j)$, soit plus légère et appartient à $\cap_{j=1}^i C(j,2-a_j)$. On a donc, afin que la solution soit valide,
\(\vert\cap_{j=1}^P C(j,a_j)\cup\cap_{j=1}^P C(j,2-a_j)\vert= 1.\) 
En fait, comme les ensembles $C(P,b)$, $b \in \{0,1,2\}$, doivent être spécifiés 
avant de connaître le résultat de la $P$-ème pesée, ils doivent être choisis de manière à ce que
 \[\vert\cap_{j=1}^{P-1} C(j,a_j) \cap C(P,b)\cup\cap_{j=1}^{P-1} C(j,2-a_j) \cap C(P,2-b)\vert\leq 1~~\forall b \in \{0,1,2\}.\] En particulier, on a 
\[\vert\cap_{j=1}^{P-1} C(j,a_j) \cap C(P,b)\vert+\vert\cap_{j=1}^{P-1} C(j,2-a_j) \cap C(P,2-b)\vert\leq 2~~\forall b \in \{0,1,2\}.\]
 Par récurrence, on démontre alors le résultat suivant pour tout $b \in \{0,1,2\}$ :
\begin{align*}\vert\cap_{j=1}^{P-i} C(j,a_j) \cap C(P+1-i,b)\cup\cap_{j=1}^{P-i} C(j,2-a_j) \cap C(P+1-i,2-b)\vert&\leq 3^{i-1},\\
  \vert\cap_{j=1}^{P-i} C(j,a_j) \cap C(P+1-i,b)\vert+\vert\cap_{j=1}^{P-i} C(j,2-a_j) \cap C(P+1-i,2-b)\vert&\leq
  3^{i-1}+1. 
  \end{align*} 
 Si le résultat est vraie pour $i$, on a
 \begin{align*}&\vert\cap_{j=1}^{P-i} C(j,a_j) \cup\cap_{j=1}^{P-i} C(j,2-a_j)\vert \\&=
 \vert\cap_{j=1}^{P-i} C(j,a_j) \cap (\cup_{b=0}^2 C(P+1-i,b))\cup\cap_{j=1}^{P-i} C(j,2-a_j) \cap (\cup_{b=0}^2 C(P+1-i,2-b))\vert
 \\&\leq \sum_{b=0}^2 \vert\cap_{j=1}^{P-i} C(j,a_j) \cap C(P+1-i,b)\cup\cap_{j=1}^{P-i} C(j,2-a_j) \cap  C(P+1-i,2-b)\vert \\&\leq 3^{i-1}+3^{i-1}+3^{i-1}=3^{i},
 \end{align*}
 et
\begin{align*}&\vert\cap_{j=1}^{P-i} C(j,a_j) \vert+\vert\cap_{j=1}^{P-i} C(j,2-a_j)\vert \\&=
 \vert\cap_{j=1}^{P-i} C(j,a_j) \cap (\cup_{b=0}^2 C(P+1-i,b))\vert+\vert\cap_{j=1}^{P-i} C(j,2-a_j) \cap (\cup_{b=0}^2 C(P+1-i,2-b))\vert
 \\&= \vert\cap_{j=1}^{P-i} C(j,a_j) \cap C(P+1-i,1)\vert+\vert\cap_{j=1}^{P-i} C(j,2-a_j) \cap  C(P+1-i,1)\vert
 \\&+ \sum_{b\in{\{0,2\}}} \vert\cap_{j=1}^{P-i} C(j,a_j) \cap C(P+1-i,b)\cup\cap_{j=1}^{P-i} C(j,2-a_j) \cap  C(P+1-i,2-b)\vert
 \\&\leq 3^{i-1}+1 + 3^{i-1} +3^{i-1}=3^{i}+1,
 \end{align*}
et comme les ensembles $C(P-i,b)$, $b \in \{0,1,2\}$, doivent être spécifiés avant de connaître le résultat de la $P-i$-ème pesée, pour tout $b \in \{0,1,2\}$,
\begin{align*}\vert\cap_{j=1}^{P-i-1} C(j,a_j) \cap C(P-i,b)\cup\cap_{j=1}^{P-i-1} C(j,2-a_j) \cap C(P-i,2-b)\vert&\leq 3^{i},\\
 \vert\cap_{j=1}^{P-i-1} C(j,a_j) \cap C(P-i,b)\vert+\vert\cap_{j=1}^{P-i-1} C(j,2-a_j) \cap C(P-i,2-b)\vert&\leq 3^{i}+1,
 \end{align*}
 donc le résultat est vraie pour $i+1$.
Pour $i=P$, on a donc
  \(\vert  C(1,b)\cup C(1,2-b)\vert\leq 3^{P-1}\) et $\vert  C(1,b)\vert+\vert C(1,2-b)\vert\leq 3^{P-1}+1,$ pour tout $b \in \{0,1,2\}$. 
En particulier, \(3^{P-1}\geq \vert  C(1,0)\cup C(1,2)\vert = \vert  C(1,0)\vert+\vert C(1,2)\vert = 2\vert C(1,2)\vert=2\vert C(1,0)\vert,\) et \(3^{P-1}+1\geq \vert  C(1,1)\vert+\vert C(1,1)\vert =2\vert C(1,1)\vert.\) Comme 
$\vert C(1,0)\vert$ et $\vert C(1,2)\vert$ sont des entiers, on a 
\[N=\vert C(1,0)\vert+\vert C(1,1)\vert+\vert C(1,2)\vert \leq \frac{3^{P-1}-1}{2} + \frac{3^{P-1}+1}{2}+\frac{3^{P-1}-1}{2}=\frac{3^{P}-1}{2}.\]
\end{enumerate}
\subsection*{Notes et références}
Un \emph{problème de balance} ou \emph{problème de pesée} est une énigme logique
où l’on dispose d’une balance à plateaux et d’un nombre limité de pesées pour
déterminer lequel parmi plusieurs objets --- souvent des pièces de monnaie ou des boules ---
a un poids différent des autres.
La solution des variantes les plus courantes est résumée dans le tableau suivant :

\begin{center}
\renewcommand{\arraystretch}{1.2} % espace vertical
\setlength{\tabcolsep}{4pt} % espace horizontal
\begin{tabularx}{\linewidth}{|X|X|c|c|}
\hline
\textbf{Hypothèse} &
\textbf{Objectif} &
\textbf{Pesées pour $N$ boules} \\
\hline
On sait si l'intrus est plus légère ou plus lourde &
Identifier l'intrus &
$P=\lceil \log_{3}(N) \rceil$ \\
\hline
L'intrus est différente (plus lourde ou plus légère) &
Identifier l'intrus &
$P=\lceil \log_{3}(2N + 1) \rceil$ \\
\hline
L'intrus est différente ou toutes sont identiques &
Identifier si l'intrus existe et si elle est plus lourde ou plus légère &
$P=\lceil \log_{3}(2N + 3) \rceil$ \\
\hline
L'intrus est différente ou toutes sont identiques, et une boule supplémentaire, dont on sait qu'elle est de poids normal, est disponible
 &
Identifier si l'intrus existe et si elle est plus lourde ou plus légère &
$P=\lceil \log_{3}(2N + 1) \rceil$ \\
\hline
L'intrus est différente ou toutes sont identiques, et une boule supplémentaire, dont on sait qu'elle est de poids normal, est disponible
 &
Identifier l'intrus &
$P=\lceil \log_{3}(2N - 1) \rceil$ \\
\hline
\end{tabularx}
\end{center}
Par exemple, avec $3$ pesées et $N=13$, il n’est pas toujours possible
de déterminer si l'intrus est plus lourde ou plus légère que les autres, mais pour $N = 12$, il est possible de trouver et de qualifier l'intrus.
Cette dernère version du problème avec $N=12$ est publiée au moins depuis 1945
et, comme le rapportent \cite{guy1995coin}, 
"elle était populaire des deux côtés de l’Atlantique durant la Seconde Guerre mondiale". La solution présentée ici est due à~\cite{dyson19461931}.
La généralisation de ce problème à plusieurs boules de poids inconnu est donnée dans~\cite{chudnov2015weighing}.


\bibliography{\currfiledir/sources.bib}
\newpage