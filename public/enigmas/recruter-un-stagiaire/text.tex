%probabilité
%optimisation
\section*{Recruter un stagiaire}
\begin{center}
    \includegraphics[width=0.4\textwidth]{\currfiledir/image.jpg}
\end{center}
\subsection*{Énoncé}
Pour recruter un stagiaire, une entreprise fait passer des entretiens individuels à quelques étudiants d'une université, un par un. Lors de chaque entretien, l’étudiant présente son bulletin de notes, indiquant sa moyenne annuelle (un score dans $[0,1]$).
À l’issue de l’entretien, l’entreprise doit immédiatement prendre une décision : accepter l’étudiant en stage (et donc terminer le processus de recrutement sans voir les autres) ou le refuser définitivement pour passer au suivant. 
En réalité, l’entreprise, peu scrupuleuse, ne se fatigue pas et regarde uniquement le score pour faire son choix.

La liste des entretiens comprend $n\geq 2$ passages (on suppose que l'entreprise connaît la valeur de~$n$). Pour chaque passage, l’étudiant a été choisi au hasard (avec remise) parmi tous les étudiants de l'université (ce qui signifie qu’un même étudiant peut passer plusieurs entretiens).

\medskip
\textbf{Questions} :
\begin{enumerate}
\item\indicators{1.6}{0.9} Peut-on concevoir une stratégie qui garantisse à l’entreprise de recruter, avec une probabilité d’au moins $1/e$, l’étudiant avec le meilleur score parmi les $n$ ?
\item\indicators{1.0}{0} L’entreprise a consulté les archives des années précédentes et connaît désormais la distribution des scores des étudiants dans l’université. 
Par ailleurs, elle apprend que l'université a suivi un protocole particulier pour constituer la liste de passage, visant à éviter qu'un excellent candidat ne soit recruté trop tôt au détriment de profils intermédiaires : l’étudiant du passage~\(i\) a été tiré au hasard parmi ceux dont le score appartient à l’intervalle \([a_i, b_i]\), où la suite \((a_i)\) est décroissante et la suite \((b_i)\) est croissante. Ces suites sont connues de l’entreprise.
Montrer que l’entreprise peut adopter une stratégie telle que le score moyen de l’étudiant recruté soit au moins égal à la moitié du score moyen du meilleur des $n$ étudiants. Peut-on faire mieux ?
\item\indicators{2.2}{0} Supposons que l’entreprise ait désormais la possibilité de réorganiser librement l’ordre des entretiens dans la liste de passage avant qu’ils ne commencent. En appliquant ensuite la stratégie optimale, quel ordre doit-elle choisir pour maximiser le score moyen de l’étudiant recruté ?

\end{enumerate}

\subsection*{Solution}
\begin{enumerate}
    \item \label{Peut-on concevoir une stratégie qui garantisse à l’entreprise de recruter} On commence par ordonner les scores des étudiants (on tranche aléatoirement les cas d’égalité). La probabilité de sélectionner le meilleur étudiant est minorée par celle de sélectionner l’étudiant de rang 1 dans cet ordre. Ainsi, on peut supposer que le meilleur étudiant est unique et uniformément réparti parmi les positions \(1\) à \(n\).
    
    On fixe un seuil \(r\), avec \(1 \leq r \leq n\). On observe et rejette systématiquement les \(r-1\) premiers étudiants. On note \(M\) le meilleur parmi ces \(r-1\) premiers. À partir de l'étudiant \(r\), on sélectionne le premier étudiant meilleur que \(M\). 
    Lorsque le meilleur étudiant occupe la position $k$, la stratégie ne le sélectionnera que si les deux conditions suivantes sont remplies :
\begin{itemize}
    \item \(k \geq r\) (il arrive après la phase d'observation),
    \item le meilleur parmi les \(k-1\) étudiants précédents se trouve parmi les \(r-1\) premiers étudiants.
\end{itemize}
La probabilité que le meilleur parmi les \(k-1\) étudiants soit dans les \(r-1\) premiers est \(\frac{r-1}{k-1}\). Ainsi, la probabilité que notre stratégie donne le meilleur étudiant vaut
\(
\sum_{k=r}^n \frac{1}{n} \cdot \frac{r-1}{k-1}.
\)

On cherche maintenant à montrer que cette probabilité est minorée par $\frac{1}{e}$.
On suppose que  $n\geq 7$, les cas restants pouvant être traités numériquement. 
On s'intéresse alors aux valeurs de \( r \) telles que \(r \geq \frac{n}{e} + 1 \).
Par la méthode des trapèzes, la probabilité est minorée par 
\(
f(r)\triangleq~\frac{r-1}{n} \left( \log\left(\frac{n-1}{r-2}\right)+\frac{1}{2}\left( \frac{1}{n-1} - \frac{1}{r-2}\right)\right),
\)
car $\log\left(\frac{n-1}{r-2}\right)=\int_{r-2}^{n-1}\frac{1}{t}dt = \sum_{k=r-1}^{n-1} \int_{k-1}^{k} \frac{1}{t}dt \leq \sum_{k=r-1}^{n-1}\frac{1}{2}\left(\frac{1}{k-1}+\frac{1}{k}\right)=\frac{1}{2}\cdot \frac{1}{r-2} + \frac{1}{r-1} + \dots + \frac{1}{n-2}+\frac{1}{2}\cdot \frac{1}{n-1}.$ 
De plus, $f$ est concave :
\begin{align*}
f'(r)&=\frac{1}{2n} \left(
\frac{1}{n - 1}
- \frac{1}{r - 2}
- \frac{(r - 1)(2r - 5)}{(r - 2)^2}
+ 2 \log\left( \frac{n - 1}{r - 2} \right)
\right)\\
f''(r)&=- \frac{r^2 - 5r + 7}{n (r - 2)^3} \leq 0.
\end{align*}
Pour $\alpha\in [0,1],$ on a alors $f\left(\frac{n}{e} + 1 + \alpha\right)\geq \min\left(f\left(\frac{n}{e} + 1\right),f\left(\frac{n}{e} + 2\right)\right)$. 
Or, comme $n\log\left(\frac{n-1}{n}\right)+{\frac{7 + (8-e)e}{2(7 + e)}}>0$ (car $x\mapsto x\log\left(\frac{x-1}{x}\right)+{\frac{7 + (8-e)e}{2(7 + e)}}$ est croissante, et positive en $x=7$), on a
\[f\left(\frac{n}{e} + 2\right)\geq P\left(\frac{1}{n}\right)\triangleq\left(\frac{1}{e}+\frac{1}{n}\right)\left(1-\frac{{\frac{7 + (8-e)e}{2(7 + e)}}}{n}+\frac{1}{2}\left(\frac{1}{n}-\frac{e}{n}\right)\right).\]
Le polynôme $P(X)-\frac{1}{e}=\frac{e}{7 + e} X (1 - 7X)$ est de degré $2$ à coefficient dominant négatif, dont les deux racines sont $0$ et $\frac{1}{7}$. Ainsi, on a bien $P\left(\frac{1}{n}\right)\geq \frac{1}{e}$. D'autre part, 
\[f(r)=\frac{r-1}{n}\int_{r-2}^{n-1}\left(\frac{1}{t} - \frac{1}{2t^2}\right)dt \geq \frac{r-1}{n}\int_{r-2}^{n-1}\frac{1}{t+1} dt=-\frac{r-1}{n}\log\left(\frac{r-1}{n}\right),\]
où l'inégalité utilise $t\geq r-2\geq 1$.
On a donc, pour $r=\frac{n}{e} + 1,$ $f(r)\geq \frac{1}{e}.$
En conclusion, on a \( f\left(\frac{n}{e} + 1 + \alpha\right) \geq \frac{1}{e} \) pour tout \( \alpha \in [0,1] \). 
Il suffit donc de choisir \( r=\lceil \frac{n}{e} + 1 \rceil  \) pour garantir une stratégie 
dont la probabilité de succès est au moins égale à \( \frac{1}{e} \).

\item \label{L’entreprise a consulté les archives des années précédentes et connaît désormais la distribution des scores des étudiants dans l’université} On considère la stratégie qui accepte le premier score \( X_i \) telle que \( X_i \geq \tau \). Soit \( r \) l'indice de ce premier score et $p\triangleq \mathbb{P} \left[\max_i X_i > \tau \right]$. Si $x\leq \tau$, alors $\mathbb{P} \left[X_r > x \right]=p.$ Sinon, $\mathbb{P} \left[X_r > x \right]=\sum_{i=1}^n \mathbb{P} \left[X_i > x \right] \prod_{j=1}^{i-1} \mathbb{P} \left[X_j \leq \tau \right] \geq \sum_{i=1}^n \mathbb{P} \left[X_i > x \right](1-p) \geq \mathbb{P} \left[\max_i X_i > x \right](1-p).$ On a donc
\(
\mathbb{E} \left[ X_r \right] = \int_0^{\infty}\mathbb{P} \left[X_r > x \right] dx \geq p\tau + \int_\tau^\infty \mathbb{P} \left[\max_i X_i > x \right](1-p)dx
\geq p\tau +(1-p)(\mathbb{E} \left[ \max_i X_i \right] - \tau)
,
\)
où la dernière inégalité vient de $\mathbb{E} \left[ \max_i X_i \right] = \int_0^\infty \mathbb{P} \left[ \max_i X_i > x \right]dx \leq \tau +\int_\tau^\infty  \mathbb{P} \left[ \max_i X_i > x \right]dx.$
En prenant \( \tau \) égal à la médiane de la distribution de \( \max_i X_i \), on a \( p = \frac{1}{2} \) et donc \(\mathbb{E}[X_r] \geq \frac{1}{2} \mathbb{E} \left[ \max_i X_i \right]\). Une autre possibilité est de choisir directement \( 
\tau = \frac{1}{2} \mathbb{E} \left[ \max_i X_i \right]. 
\)

La constante \( \frac{1}{2} \) est optimale. En effet, soit \( \varepsilon > 0 \). Considérons une distribution des scores dans l’université donnée par :
\[
\begin{cases}
0 & \text{avec probabilité } (1 - \varepsilon)^2, \\
\varepsilon & \text{avec probabilité } \varepsilon, \\
1 & \text{avec probabilité } (1 - \varepsilon)\varepsilon.
\end{cases}
\]
Prenons ensuite \( n = 2 \), avec \( a_1 = b_1 = \varepsilon \) et \( a_2 = 0 \), \( b_2 = 1 \). On a alors \( X_1 =  \mathbb{E}[X_2] = \varepsilon \). Ainsi, quelle que soit la stratégie utilisée, le score moyen de l'étudiant recruté est toujours égal à \( \varepsilon \), tandis que 
\(
\mathbb{E}[\max(X_1, X_2)] = \varepsilon(1 + (1 - \varepsilon)^2).
\)
Le rapport est donc de
\(
\frac{1}{1 + (1 - \varepsilon)^2}\to_{\varepsilon \to 0} \frac{1}{2}.
\)

\item \label{Supposons que l’entreprise ait désormais la possibilité de réorganiser librement l’ordre des entretiens dans la liste de passage avant qu’ils ne commencent} On considère \( V(X_1,\dots,X_n) \) comme étant le score espéré de la stratégie optimale (en ligne). Par programmation dynamique, on a la relation de récurrence
\[
V(X_1, \dots, X_n) = \mathbb{E}\left[\max(X_1, V(X_2, \dots, X_n))\right].
\]
On souhaite montrer que l’ordre optimal est l’ordre inverse, c’est-à-dire \( X_n, \dots, X_1 \). Comme toute permutation peut s’écrire comme une composition de transpositions d’éléments consécutifs, il suffit de montrer que :
\[
V(X_{k_1}, \dots, X_{k_n}) 
\geq 
V(X_{k_1}, \dots, X_{k_{j-1}}, X_{k_{j+1}}, X_{k_j}, X_{k_{j+2}}, \dots, X_{k_n}),
\]
pour un certain \( j \) avec \( k_j \geq k_{j+1}.\) D’après la formule de programmation dynamique, cela revient à prouver que :
\[
V(X_{k_j}, X_{k_{j+1}}, c) \geq V(X_{k_{j+1}}, X_{k_j}, c),
\]
où \( c = V(X_{k_{j+2}}, \dots, X_{k_n}) \).
\begin{definition}
   Deux variables aléatoires \( Y \) et \( Z \) vérifient la relation \( Y \succ Z \) si, pour tout \( c \in [0,1] \), on a \( 
V(Y, Z, c) \geq V(Z, Y, c). 
\)  
\end{definition}
\begin{remark}
Si \( Y \) et \( Z \) sont deux variables aléatoires dont les supports sont des intervalles disjoints, alors \( Y \succ Z \).
\end{remark}
\begin{lemma}
Soient \( Y, Y' \) et \( Z \) trois variables aléatoires telles que \( Y \succ Z \) et \( Y' \succ Z \).  
Alors, toute variable aléatoire \( M(Y,Y') \) définie comme un mélange de \( Y \) et \( Y' \), c'est-à-dire  
\[
M(Y,Y') \sim B Y + (1 - B) Y' \quad \text{avec } B \text{ une variable de Bernoulli indépendante de } Y, Y',
\]
vérifie également \( M(Y,Y') \succ Z \).
\end{lemma}
\begin{proof}
Soit \( \lambda \triangleq \mathbb{E}[B] \). La fonction \( V(Z, \cdot) = \mathbb{E}[\max(Z, \cdot)] \) est convexe, donc :

\[
\begin{aligned}
V(Z, M(Y,Y'), c) &= V(Z, V(M(Y,Y'), c)) \\
&= V\left(Z, \lambda V(Y, c) + (1 - \lambda)V(Y', c)\right) \\
&\leq \lambda V(Z, V(Y, c)) + (1 - \lambda)V(Z, V(Y', c)) \\
&= \lambda V(Z, Y, c) + (1 - \lambda)V(Z, Y', c) \\
&\leq \lambda V(Y, Z, c) + (1 - \lambda)V(Y', Z, c) \\
&= V(M(Y,Y'), Z, c).
\end{aligned}
\]
\end{proof}

D'après le contexte, pour tout \( k \in \{1, \dots, n\} \), la variable aléatoire \( X_k \) suit la distribution des scores de l'université, conditionnée au fait de prendre ses valeurs dans l'intervalle \( [a_k, b_k] \). 
Comme les intervalles sont imbriqués, la loi de \( X_{k_j} \), conditionnellement à \( X_{k_j} \in [a_{k_{j+1}}, b_{k_{j+1}}] \), est exactement celle de \( X_{k_{j+1}} \). Par conséquent, la loi de \( X_{k_j} \) peut être vue comme un mélange de trois lois : une variable \( Y \) de support \( [a_{k_j}, a_{k_{j+1}}) \), la variable \( X_{k_{j+1}} \), et une variable \( Z \) de support \( (b_{k_{j+1}}, b_{k_j}] \). On note $M(Y,X_{k_{j+1}},Z)$ ce mélange.
On a clairement \( X_{k_{j+1}} \succ X_{k_{j+1}} \).  
D’après la remarque précédente, on a également \( Y \succ X_{k_{j+1}} \) et \( Z \succ X_{k_{j+1}} \).  
En appliquant deux fois le lemme précédent (pour faire un mélange de trois lois), on en déduit que \(X_{k_j} \sim M(Y,X_{k_{j+1}},Z) \succ X_{k_{j+1}} \).








\end{enumerate}
\subsection*{Notes et références}
L’énigme s’inscrit dans le cadre d’un ancien problème classique, déjà évoqué par Cayley~\cite{Cay:ET1875} et formalisé par Moser~\cite{Mos:SM1956}. Il s’agit d’un problème d’arrêt optimal sur une suite de variables aléatoires indépendantes et identiquement distribuées, où l’on observe séquentiellement les réalisations \( X_1, \dots, X_n \), et où l’on doit décider à quel moment s’arrêter définitivement.

La question~\ref{Peut-on concevoir une stratégie qui garantisse à l’entreprise de recruter} correspond à ce qu’on appelle couramment le \emph{problème des secrétaires}~\cite{gilbert1966recognizing}.

La question~\ref{L’entreprise a consulté les archives des années précédentes et connaît désormais la distribution des scores des étudiants dans l’université} renvoie à la célèbre \emph{inégalité du prophète}~\cite{krengel1978semiamarts, hill1992survey}, qui compare les performances d’un décideur "en ligne" à celles d’un "prophète" qui connaîtrait toutes les valeurs à l’avance. 

Enfin, la question~\ref{Supposons que l’entreprise ait désormais la possibilité de réorganiser librement l’ordre des entretiens dans la liste de passage avant qu’ils ne commencent} explore la problématique de l’\emph{ordre optimal de sélection} dans les problèmes d’arrêt séquentiel, étudiée notamment par~\cite{hill1985selection}.

\bibliography{\currfiledir/sources.bib}
\newpage