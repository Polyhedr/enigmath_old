%géométrie
%optimisation
\section{Quadrivillage}
\begin{center}
    \includegraphics[width=0.4\textwidth]{enigmes/Quadrivillage/image.png}
\end{center}
\subsection*{Énoncé}
Le village de Quadrivillage est organisé selon un quadrillage de rues.  
À certaines intersections se trouve une maison, pour un total de $n$ maisons dans le village (chaque maison abrite exactement un habitant).  
Le maire souhaite construire une Grande Allée : une route traversant le village en ligne droite, de façon oblique.  
Son objectif est de placer cette route de manière à réduire au maximum le temps de trajet des habitants pour s’y rendre.

\medskip
\textbf{Questions} :
\begin{enumerate}
\item\indicators{0.4}{0} En supposant que les habitants se déplacent uniquement le long des rues du quadrillage, 
déterminez, pour une Grande Allée donnée, le temps de trajet d'un habitant jusqu'à celle-ci. 
\item\indicators{2.1}{0} Proposez un algorithme en temps  quasi-quadratique en $n$, 
permettant de déterminer la position optimale de la Grande Allée, 
de manière à minimiser la somme des temps de trajet de tous les habitants 
jusqu’à celle-ci.
\item \indicators{2.5}{0} Avant le début de la construction, plusieurs villages voisins (organisés eux aussi en quadrillage) souhaitent 
bâtir leur Grande Allée selon le même objectif. Cependant, selon la tradition locale, toutes ces Grandes Allées doivent  
être parallèles entre elles. Les maires se réunissent donc pour déterminer conjointement la position optimale 
de chaque Grande Allée. Si l'on considère $d$ villages et $n$ habitants par village, décrivez un algorithme  
de complexité $\mathcal O(dn^2\log(dn))$ permettant de résoudre ce problème de minimisation. 
\end{enumerate}
\subsection*{Solution}
\begin{enumerate}
\item La Grande Allée est modélisée par une droite d’équation $y = ax + b$, avec $(a,b)\in \mathbb{R}^2$.  
Si l’on note $(x_i, y_i)$ la position de la $i$-ème maison, le temps de trajet jusqu’à la route est donné par la plus petite distance horizontale ou verticale :  
\[
\min\big(\lvert ax_i + b - y_i \rvert , \lvert x_i - a^{-1}(y_i - b) \rvert \big)
=\lvert ax_i + b - y_i \rvert \cdot \min\big(1,\lvert a\rvert^{-1}\big).
\]
    \item\label{Proposez un algorithme en temps  quasi-quadratique} 
Le problème global se ramène à la minimisation suivante :
\[
\min_{a,b}
\sum_{i=1}^n \lvert ax_i+b-y_i\rvert \cdot \min\big(1,\lvert a\rvert^{-1}\big)
=
  \min\Bigg( \min_{a,b}\sum_{i=1}^n \lvert ax_i+b-y_i\rvert , 
\min_{a,b}\sum_{i=1}^n \big\lvert x_i - a^{-1}(y_i-b) \big\rvert \Bigg).
\]

Autrement dit, il suffit de résoudre les deux cas (tous les habitants se raccordent horizontalement ou tous les habitants se raccordent verticalement) et de retenir le plus favorable.  
De plus, en échangeant le rôle des $x_i$ et des $y_i$, et en effectuant le changement de variables $(a,b) \mapsto (1/a,\,-b/a)$, le second problème se ramène au premier.  
En conclusion, on veut déterminer la droite d’équation $y=ax+b$ qui minimise :
\[
\min_{a,b} \sum_{i=1}^n \lvert ax_i+b-y_i\rvert.
\]
Pour ce faire, introduisons d’abord la notion de médiane.
\begin{definition}
    Soient $(x_j) \in \mathbb{R}^n$ et $(w_j) \in \mathbb{R}^n_+$.  
    Une médiane de $(x_j)$ avec poids $(w_j)$ est tout réel
    \[
        m \in \arg\min_{t \in \mathbb{R}} \sum_{j=1}^n w_j \, |x_j - t|.
    \]
\end{definition}
\begin{lemma}
    Soient $(x_j) \in \mathbb{R}^n$ et $(w_j) \in \mathbb{R}^n_+$.  
    Alors il existe un indice $i_0 \in \{1,\dots,n\}$ tel que $x_{i_0}$ soit une médiane de $(x_j)$ avec poids $(w_j)$.
\end{lemma}
\begin{proof}
 Définissons la fonction objectif
    \[
        f(t) \triangleq \sum_{j=1}^n w_j \, |x_j - t|, \qquad t \in \mathbb{R}.
    \]
Supposons que les $x_j$ soient ordonnés :
    \(
        x_1 \leq x_2 \leq \cdots \leq x_n.
    \)
    Posons $x_0 \triangleq -\infty$ et $x_{n+1}\triangleq+\infty$.  
    Pour $t \in [x_i, x_{i+1}]$ avec $i \in \{0,\dots,n\}$, on a
    \[
        f(t) = \sum_{j=1}^i w_j (t - x_j) + \sum_{j=i+1}^n w_j (x_j - t).
    \]
    C’est une fonction affine en $t$ sur l’intervalle $[x_i,x_{i+1}]$, de pente
    $
        g_i \triangleq \sum_{j=1}^i w_j - \sum_{j=i+1}^n w_j.
    $
    La suite $(g_i)$ est croissante en $i$ et il existe $i_0\in \{1,\dots,n\}$ tel que
    \(
        g_{i_0-1} \leq 0 \leq g_{i_0}.
    \)
    Ainsi, $f$ est décroissante sur $(-\infty, x_{i_0})$ et croissante sur $(x_{i_0},+\infty)$, 
    donc $f$ atteint un minimum en $x_{i_0}$.  
\end{proof}
\begin{remark}
    Le calcul d’une médiane et de la valeur de la fonction objectif associée peut s’effectuer en temps quasi-linéaire en $n$, 
    en triant les $x_i$ puis en cherchant l’indice $i$ où $g_i$ change de signe (le calcul de tous les $g_i$ et l’évaluation de la fonction objectif sont linéaires en $n$).
\end{remark}

Pour déterminer la pente optimale \(a\), on procède ainsi :  
pour chaque $i_0$, on considère les rapports 
\(\tfrac{y_j - y_{i_0}}{x_j - x_{i_0}}\) pondérés par 
\(\lvert x_j - x_{i_0}\rvert\).  
On applique ensuite l’algorithme de calcul de médiane à cette famille, ce qui fournit 
une médiane \(m_{i_0}\) ainsi que la valeur correspondante de la fonction objectif, notée \(v_{i_0}\). 
On choisit \(a\) en minimisant sur $i_0$, soit
\(
a \triangleq m_{i_0^*} \) avec \(v_{i_0^*} \triangleq \min_{i_0} v_{i_0},
\)
ce qui conduit à une complexité quasi-quadratique. Une fois \(a\) choisi, on déduit le paramètre d'ordonnée à l'origine par  
\(
b \triangleq y_{i_0^*} - a x_{i_0^*}.
\)
On peut vérifier l’optimalité du couple \((a,b)\) de la façon suivante :
\begin{align*}
\min_{a,b} \sum_{i=1}^n \lvert ax_i+b-y_i\rvert &= \min_a\min_{i_0\in \{1,\dots,n\}} \sum_{i=1}^n \lvert a(x_i-x_{i_0})+y_{i_0}-y_i\rvert
\\&=
\min_{i_0\in \{1,\dots,n\}}\min_a \sum_{i=1}^n \vert x_i-x_{i_0}\vert\cdot\lvert a-\frac{y_i-y_{i_0}}{x_i-x_{i_0}}\rvert.
\end{align*}

\item \label{Avant le début de la construction} Par un raisonnement analogue à celui de la question \ref{Proposez un algorithme en temps  quasi-quadratique}, il suffit de résoudre 
\[
\min_{a,b_1,\dots,b_d}
\sum_{k=1}^d\sum_{i=1}^n \lvert ax_{ki}+b_k-y_{ki}\rvert.
\]
Comme précédemment, on sait que chaque $b_k$ optimal est de la forme 
\[
b_k = y_{ki_k}-ax_{ki_k},
\]
pour un certain $i_k$.  
Cependant, contrairement au cas unidimensionnel, on ne peut pas parcourir exhaustivement tous les choix possibles de $(i_1,\dots,i_d)$, car cela mènerait à une complexité prohibitive de l’ordre de $n^d$.  
En revanche, on sait que l’optimum en $a$ doit apparaître pour une valeur de la forme
\[
a=\frac{y_{kj}-y_{ki_k}}{x_{kj}-x_{ki_k}}.
\]
Il y a au plus $d n^2$ tels rapports candidats.  
On commence par trier les ratios candidats, pour un coût de $\mathcal O(d n^2 \log(d n))$. 
On peut ensuite effectuer une recherche binaire sur ces ratios afin de déterminer le $a$ optimal. 
Grâce au lemme suivant, cette recherche est possible : à chaque étape, on évalue la fonction et sa pente pour décider de quel côté se situe le minimum (c'est-à-dire le changement de signe de la pente). 
Le coût total de cette recherche binaire est $\mathcal O(\log(d n) \cdot d n \log n)$. 
Comme cette complexité est inférieure à celle du tri initial, le résultat global est préservé.

\begin{lemma}
La fonction
\[
a \;\mapsto\; \min_{b_1,\dots,b_d} \sum_{k=1}^d \sum_{i=1}^n \lvert a x_{ki} + b_k - y_{ki} \rvert
\]
est convexe et affine sur chaque intervalle ouvert délimité par deux ratios consécutifs. 
L'évaluation de la fonction ainsi que le calcul de sa pente se fait en temps $\mathcal O(d n \log n)$.
\end{lemma}

\begin{proof}
La convexité provient du fait qu'il s'agit du minimum partiel d'une fonction conjointe convexe en $(a, b_1, \dots, b_d)$.

Pour un $k$ fixé et pour tout $a$ dans un intervalle ouvert délimité par deux ratios consécutifs, l'ordre des valeurs $a x_{ki} + b_k - y_{ki}$ reste constant. 
Ainsi, le $b_k$ optimal reste de la forme $b_k = y_{ki_k} - a x_{ki_k}$ avec un indice $i_k$ fixe sur tout l'intervalle, et la fonction est donc linéaire en $a$ sur cet intervalle.

Enfin, pour chaque candidat $a$, les $b_k$ optimaux sont obtenus via $d$ calculs de médiane, chacun coûtant $\mathcal O(n \log n)$, ce qui donne un coût total $\mathcal O(d n \log n)$. 
Une fois les $b_k$ déterminés, l'évaluation de la fonction et de sa pente se fait en temps $\mathcal O(d n)$.
\end{proof}



\end{enumerate}


\subsection*{Notes et références}
Cette énigme s'articule autour de la \emph{méthode des moindres écarts absolus} (LAD, \emph{Least Absolute Deviations}), qui consiste à minimiser la somme des écarts absolus entre un modèle et des données, soit la norme $L_1$ des résidus. Cette méthode est robuste aux valeurs aberrantes et correspond au maximum de vraisemblance si les erreurs suivent une loi de Laplace \cite{boscovich1757}.

Comme on l'a vu, ajuster une droite à un ensemble de points bidimensionnels $(x_i,y_i)$ fait intervenir naturellement le concept de médiane. Un estimateur particulièrement pertinent dans ce contexte est l'\emph{estimateur de Theil--Sen} \cite{sen1968estimates}, qui choisit la pente $m$ comme la médiane des pentes $(y_j - y_i)/(x_j - x_i)$ entre toutes les paires de points. Une fois la pente fixée, l'ordonnée à l'origine optimale peut être prise comme la médiane des valeurs $y_i - m x_i$.

Ce type de régression robuste trouve également des applications en vision 3D. Par exemple, pour calibrer les paramètres intrinsèques d'une caméra à partir d'un nuage de points 3D $(X_i,Y_i,Z_i)$ et de leurs projections dans l'image $(u_i,v_i)$, on cherche à estimer la focale $f$ et le centre optique $(c_x,c_y)$. La projection est alors de la forme
\[
u_i \approx f \frac{X_i}{Z_i} + c_x, \quad v_i \approx f \frac{Y_i}{Z_i} + c_y.
\]
En considérant une loss $L_1$, ce problème se rapproche de la structure de la question~\ref{Avant le début de la construction} avec $d=2$. Plus récemment, la même approche est utilisé pour réaligner les prédictions d'un nuage de points 3D avec la vérité terrain correspondante, afin d'obtenir une supervision plus efficace \cite{wang2025moge}.





\newpage