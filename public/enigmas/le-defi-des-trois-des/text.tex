%probabilité
%théorie des jeux
\section*{Le défi des trois dés}
\begin{center}
    \includegraphics[width=0.4\textwidth]{\currfiledir/image.png}
\end{center}
\subsection*{Énoncé}
Considérons un jeu auquel participent \(K\) joueurs. Chaque joueur reçoit trois dés à six faces. Chacun lance simultanément ses trois dés et obtient un score,
    défini comme la somme des trois résultats. Après ce premier lancer, chaque joueur doit décider :
    \begin{itemize}
        \item soit de conserver ce score et d'arrêter son tour,
        \item soit de relancer les trois dés pour tenter d'obtenir
        un meilleur score.
    \end{itemize}
    Si un joueur choisit de relancer, il perd irréversiblement son score précédent,
même si celui-ci était plus élevé. Chaque joueur peut relancer au maximum deux fois, 
c'est-à-dire qu'il effectue entre un et trois lancers au total. 
Les scores intermédiaires sont gardés secrets jusqu'à la fin de la partie. Lorsque tous les joueurs ont terminé, les scores sont révélés. 
Le joueur qui possède le plus petit score est déclaré perdant. 
En cas d'égalité pour le plus faible score, le perdant est choisi au hasard 
parmi les ex~æquo.

\medskip
\textbf{Questions} :
\begin{enumerate}
\item\indicators{2.1}{2.9} Déterminez, en fonction de $K$, un profil de stratégies pures 
(c’est-à-dire non aléatoires) tel que chaque joueur adopte 
sa meilleure réponse compte tenu des choix des autres joueurs.
\item\indicators{2.2}{0} On considère un joueur isolé face à une coalition composée des $K-1$ autres joueurs, et on s’intéresse aux stratégies mixtes (le coup joué est sélectionné au hasard selon une certaine distribution de probabilité). Montrez que, sans perte de généralité, le problème peut se ramener à un jeu à somme nulle de taille \(120\times120\) entre deux joueurs (le joueur isolé versus la coalition).
\item\indicators{2.3}{0} En généralisant à $\ell+1$ lancers, montrez qu'un joueur isolé face à une coalition composée des $K-1$ autres joueurs peut se ramener à un jeu à somme nulle 
de dimension $\binom{14 + \ell}{\ell} \times \binom{14 + \ell}{\ell}$ entre deux joueurs.
\item\indicators{1.1}{3.3} On considère de nouveau le cas du jeu en concurrence pure (sans coalition). On suppose qu’après chaque relance, tous les joueurs observent le nombre $L$ de joueurs encore en jeu (n’ayant pas arrêté). Pour trois lancers et pour tout $K\le 100$, déterminez un profil de stratégies pures tel que chaque joueur adopte sa meilleure réponse compte tenu des décisions des autres.
\end{enumerate}


\subsection*{Solution}
\begin{enumerate}
    \item\label{Déterminer, en fonction de $K$} On note $p_s$ la probabilité qu'un lancer de trois dés donne un score total égal à $s$.  
Une \emph{stratégie pure de seuil} est définie par deux entiers \(t_1, t_2 \in \{4, \dots, 18\}\) tels que :
\begin{itemize}
  \item après le premier lancer donnant le score \(s\), le joueur \emph{conserve} son score si \(s \geq t_1\), et \emph{relance} sinon ;
  \item si le joueur relance une première fois et obtient un nouveau score \(s'\), il conserve ce score si \(s' \geq t_2\), et relance une dernière fois sinon.
\end{itemize}

On suppose dans un premier temps que \(K \geq 2\,887\,502\).  
Pour la stratégie pure définie par les seuils \((4,4)\), on note \(X\) la variable aléatoire représentant 
le score final obtenu après application de cette stratégie.  

Nous allons montrer que la stratégie pure \((4,4)\) constitue une \emph{meilleure réponse} pour un joueur donné, en supposant que tous les adversaires jouent également la stratégie pure \((4,4)\), autrement dit que le profil où tous les joueurs jouent la stratégie \((4,4)\) forme un \emph{équilibre de Nash}.
La probabilité que le joueur considéré \emph{perde}, sachant que son score final vaut \(s \in \{3, \dots, 18\}\), est :
\begin{align*}
f_X(s)
&\triangleq 
\sum_{j=0}^{K-1} \binom{K-1}{j} 
\mathbb{P}(X = s)^j 
\mathbb{P}(X > s)^{K-1-j} 
\frac{1}{j+1} \\
&= 
\frac{\mathbb{P}(X \geq s)^K - \mathbb{P}(X > s)^K}{K\,\mathbb{P}(X = s)} 
= \frac{1}{K} \sum_{j=0}^{K-1} \mathbb{P}(X > s)^j \mathbb{P}(X \geq s)^{K-1-j}
.
\end{align*}
Cette dernière expression montre que \(f_X(s)\) est une fonction décroissante de \(s\). 
Pour qu'une stratégie pure \((t_1, t_2)\) soit une meilleure réponse, 
il faut que chaque seuil \(t_i\) vérifie :
\[
t_i \in 
\arg\min_{t} 
\left( 
\sum_{s = t}^{18} p_s f_X(s) 
+ 
\sum_{s = 3}^{t-1} p_s E_i 
\right),
\]
où les quantités \(E_1\) et \(E_2\) sont définies par :
\[
E_2 = \sum_{s=3}^{18} p_s f_X(s),
\qquad
E_1 = \sum_{s = t_2}^{18} p_s f_X(s) 
+ 
\sum_{s = 3}^{t_2 - 1} p_s E_2.
\]
Autrement dit, le seuil optimal peut se réécrire sous la forme :
\[
t_i =\min \{\, s \mid f_X(s) \leq E_i \,\}= \max \{\, s \mid f_X(s-1) > E_i \,\},
\]
ce maximum existant puisque \(f_X(3) > E_i\) pour tout \(i \in \{1,2\}\).
Pour que la stratégie pure \((4,4)\) soit une meilleure réponse, il faut donc que $f_X(4)\leq E_i$ pour tout \(i \in \{1,2\}\), i.e.,
  \[
  f_X(4) \leq \sum_{s=3}^{18} p_s f_X(s),
 \quad
  f_X(4) \leq \sum_{s = 4}^{18} p_s f_X(s) + p_3 \sum_{s=3}^{18} p_s f_X(s).
  \]

Ces deux conditions sont satisfaites si l'on montre que \(f_X(4) \leq p_3^2 f_X(3)\).
Or, on a :
\[
\mathbb{P}(X \geq 4) = 1 - p_3^3
\quad \text{et} \quad
\mathbb{P}(X > 4) = (1 - p_3^3) \frac{1 - p_3 - p_4}{1 - p_3}.
\]
Ainsi :
\[
f_X(4) = 
\frac{(1 - p_3^3)^K 
\left( 1 - \left( \frac{1 - p_3 - p_4}{1 - p_3} \right)^K \right)}
{K p_4 \frac{1 - p_3^3}{1 - p_3}},
\qquad
f_X(3) = 
\frac{1 - (1 - p_3^3)^K}{K p_3^3}.
\]

On en déduit :
\[
\frac{f_X(4)}{p_3^2 f_X(3)} 
= 
\frac{(1 - p_3^3)^K \left( 1 - \left( \tfrac{1 - p_3 - p_4}{1 - p_3} \right)^K \right) p_3}
{p_4 \tfrac{1 - p_3^3}{1 - p_3} \left( 1 - (1 - p_3^3)^K \right)} 
\leq 
\frac{(1 - p_3^3)^K p_3}
{p_4 \tfrac{1 - p_3^3}{1 - p_3} \left( 1 - (1 - p_3^3)^K \right)}.
\]
Cette quantité est inférieure à \(1\) dès que
\[
(1 - p_3^3)^{-K} \geq 
 1 + 
\frac{p_3}{p_4 \frac{1 - p_3^3}{1 - p_3}},
\quad \text{c'est-à-dire} \quad
K \geq 
-\frac{
\log\!\left( 
1 + 
\frac{p_3}{p_4 \frac{1 - p_3^3}{1 - p_3}} 
\right)
}{
\log(1 - p_3^3)
}
\approx 2\,887\,501{.}82,
\]
ce qui est bien vérifié par hypothèse.

Pour \(K \leq 2\,887\,501\), nous utilisons le code Python suivant, permettant de déterminer un équilibre de Nash symétrique pour chaque $K$.  
Sur un ordinateur portable classique, le calcul complet nécessite environ \(400\) secondes. Les équilibres obtenus sont résumés dans le tableau ci-dessous.  
Dans le code, on cherche un point fixe d’une certaine application en itérant ses compositions successives, 
en prenant comme stratégie initiale \((4,4)\).  
L’unicité de l’équilibre de Nash n’est pas garantie : 
plusieurs profils de seuils peuvent simultanément constituer des équilibres pour certaines valeurs de \(K\).  
Cependant, le choix de la stratégie initiale \((4,4)\) assure une continuité entre les deux régimes 
\(K \leq 2\,887\,501\) et \(K \geq 2\,887\,502\) : dès que \((4,4)\) devient un équilibre, 
c’est celui qui est effectivement sélectionné par l’algorithme.


\begin{center}
\renewcommand{\arraystretch}{1.2}
\begin{tabular}{c|c}
\textbf{Intervalle de \(K\)} & \textbf{Seuils d'équilibre \((t_1, t_2)\)} \\
\hline
$2 > K \geq 0$ & non défini \\
$3 > K \geq 2$ & $(13, 12)$ \\
$4 > K \geq 3$ & $(12, 12)$ \\
$8 > K \geq 4$ & $(12, 11)$ \\
$11 > K \geq 8$ & $(11, 11)$ \\
$18 > K \geq 11$ & $(11, 10)$ \\
$34 > K \geq 18$ & $(10, 10)$ \\
$51 > K \geq 34$ & $(10, 9)$ \\
$120 > K \geq 51$ & $(9, 9)$ \\
$190 > K \geq 120$ & $(9, 8)$ \\
$546 > K \geq 190$ & $(8, 8)$ \\
$946 > K \geq 546$ & $(8, 7)$ \\
$3\,265 > K \geq 946$ & $(7, 7)$ \\
$6\,509 > K \geq 3\,265$ & $(7, 6)$ \\
$31\,019 > K \geq 6\,509$ & $(6, 6)$ \\
$77\,495 > K \geq 31\,019$ & $(6, 5)$ \\
$713\,156 > K \geq 77\,495$ & $(5, 5)$ \\
$2\,852\,409 > K \geq 713\,156$ & $(5, 4)$ \\
$\infty > K \geq 2\,852\,409$ & $(4, 4)$ \\
\end{tabular}
\end{center}

\begin{lstlisting}
(*@\codeheader{\currfiledir/1.py}@*)
(*@ @*)
\end{lstlisting}
\vspace{-.455cm}
\lstinputlisting{\currfiledir/1.py}
    
    \item\label{On s’intéresse à la stratégie} La coalition est représentée par les \(K-1\) premiers joueurs, et le joueur "seul" est le joueur \(K\), dont on examine la probabilité de perdre. On note $X_i$ la variable aléatoire représentant le score final obtenu par le joueur $i\in \{1,\dots,K-1\}.$ La probabilité que le joueur seul perde, sachant que son score final vaut \(s \in \{3, \dots, 18\}\), est :
\[\sum_{S\subseteq \{1,\dots,K-1\}} 
\prod_{i\in S}\mathbb{P}(X_i = s)\prod_{i\notin S}
\mathbb{P}(X_i > s) 
\frac{1}{\vert S\vert+1}.\]
Si l’on introduit une variable aléatoire \(U \sim \mathcal{U}(0,1)\), alors \(\mathbb{E}(U^k) = \frac{1}{k+1}\) pour tout \(k \in \mathbb{N}\). La probabilité précédente peut donc se réécrire sous la forme :
    \begin{align*}\mathbb{E}\left(\sum_{S\subseteq \{1,\dots,K-1\}} 
\prod_{i\in S}\mathbb{P}(X_i = s)\prod_{i\notin S}
\mathbb{P}(X_i > s) 
U^{\vert S \vert}\right) &= \mathbb{E}\left(\prod_{i=1}^{K-1}\left(\mathbb{P}(X_i = s)U+\mathbb{P}(X_i > s)\right)\right)
%\\&= \mathbb{E}\left(\prod_{i=1}^{K-1}\left(\mathbb{P}(X_i \ge s)U+\mathbb{P}(X_i > s)(1-U)\right)\right)
%\\&\le \mathbb{E}\left(\frac{1}{K-1}\sum_{i=1}^{K-1}\left(\mathbb{P}(X_i \ge s)U+\mathbb{P}(X_i > s)(1-U)\right)^{K-1}\right)
.\end{align*}
    Par l’inégalité arithmético-géométrique, on obtient la borne supérieure suivante :
    \[\mathbb{E}\left(\frac{1}{K-1}\sum_{i=1}^{K-1}\left(\mathbb{P}(X_i = s)U+\mathbb{P}(X_i > s)\right)^{K-1}\right)=\frac{1}{K-1}\sum_{i=1}^{K-1} f_{X_i}(s).\]
    Ainsi, du point de vue de la coalition, il est toujours au moins aussi avantageux de jouer la
\emph{stratégie mixte symétrique}
dans laquelle chaque joueur de la coalition adopte les mêmes seuils, choisis uniformément parmi les paires
\(\{(t_{i,1}, t_{i,2}) \mid i \in {1, \dots, K-1}\}\),
plutôt qu’une
\emph{stratégie pure hétérogène}
où chaque joueur $i$ aurait ses propres seuils \((t_{i,1}, t_{i,2})\).
En conclusion, lorsqu’on énumère les stratégies pures envisageables pour la coalition, on peut, sans perte de généralité,
se restreindre aux \emph{profils de coalition symétriques}.

\medskip
On note \(X_{\tau}\) la variable aléatoire correspondant au score final obtenu lorsqu’un joueur applique la stratégie \(\tau=(\tau_1,\tau_2)\).
On peut alors se restreindre à un jeu à somme nulle à deux joueurs : le joueur "seul" contre la coalition.
Lorsque la coalition adopte la stratégie pure \(t=(t_1,t_2)\) (i.e., la stratégie pure symétrique \((t_1,t_2),\dots,(t_1,t_2)\)) et que le joueur seul applique la stratégie pure \(\hat t =(\hat t_1,\hat t_2)\),
la valeur du jeu est donnée par
\(
\mathbb{E}\left(f_{X_{t}}(X_{\hat t})\right).
\)
On remarque que, conformément à l’intuition, cette quantité croît lorsque la loi de la coalition devient plus favorable,
c’est-à-dire lorsque \(\mathbb{P}(X_{t} \geq \cdot)\) augmente.
De même, cette quantité décroît lorsque la loi du joueur seul devient plus favorable, c’est-à-dire lorsque \(\mathbb{P}(X_{\hat t} \geq \cdot)\) augmente.   
On peut donc éliminer une stratégie pure \(\tau=(\tau_1, \tau_2)\) envisagée par l’un ou l’autre des deux camps dès que la fonction de survie associée \(\mathbb{P}(X_{\tau} \geq \cdot)\) est majorée en tout point par une combinaison convexe de deux autres fonctions de survie : 
\[
\mathbb{P}(X_{\tau} \geq \cdot)
\leq
\alpha \mathbb{P}(X_{\tau'} \geq \cdot)
+
(1-\alpha) \mathbb{P}(X_{\tau''} \geq \cdot), 
\qquad \text{avec } \alpha \in [0,1].
\]
On note $\tau \prec [\tau',\tau'']$ si c'est le cas, et on dit que la stratégie \(\tau\) est dominée par une stratégie mixte dont le support est formé des deux stratégies pures 
\(\tau'\) et \(\tau''\). Le lemme suivant permet de réaliser une telle élimination.

\begin{lemma}
   Pour $\tau_1<\tau_2$, on a $(\tau_1, \tau_2) \prec [(\tau_1, \tau_1),(\tau_2, \tau_2)].$
Autrement dit, il suffit de considérer les stratégies pures de la forme  \((\tau_1, \tau_2)\) vérifiant \(\tau_1 \geq \tau_2\).
Cette restriction est naturelle : lors du premier choix, il reste encore deux lancers possibles, ce qui permet d’adopter une attitude plus risquée, autrement dit, de fixer un seuil plus élevé qu'au second lancer.
\end{lemma}    
\begin{proof}
On pose
\(
R(x)\triangleq\sum_{s\ge x} p_s,
\) $A=1-R(\tau_1),~B=R(\tau_1)-R(\tau_2),~C=R(\tau_2)$ et \(\displaystyle \alpha=\frac{A+B}{2A+B}\). Les expressions des probabilités
de dépasser \(s\) sont alors les suivantes, 
avec $E(s)$ la probabilité que le score final soit supérieur à $s$, sachant qu'on a effectué deux relances (contre $\tau_1$ et $\tau_2$). Dans le cas présent, on a $E(s) = R(s)$, mais cette égalité ne sera plus vérifiée dans la question suivante.
\begin{align*}
\mathbb{P}(X_{(\tau_1,\tau_2)}\ge s)
&= R(\max(s,\tau_1)) \;+\; A\Big( R(\max(s,\tau_2)) + (A+B)\,E(s)\Big),\\[4pt]
\mathbb{P}(X_{(\tau_1,\tau_1)}\ge s)
&= R(\max(s,\tau_1)) \;+\; A\Big( R(\max(s,\tau_1)) + A\,E(s)\Big),\\[4pt]
\mathbb{P}(X_{(\tau_2,\tau_2)}\ge s)
&= R(\max(s,\tau_2)) \;+\; (A+B)\Big( R(\max(s,\tau_2)) + (A+B)\,E(s)\Big).
\end{align*}
On pose 
pour tout \(s\),
\[
T(s)\;\triangleq\;\mathbb{P}(X_{(\tau_1,\tau_2)}\ge s)
- \alpha\,\mathbb{P}(X_{(\tau_1,\tau_1)}\ge s) - (1-\alpha)\,\mathbb{P}(X_{(\tau_2,\tau_2)}\ge s).\]
On observe que les termes en $E(s)$ se compensent exactement dans l'expression de $T(s)$. Afin de montrer que $T(s) \le 0$, on distingue trois cas :

\textbf{Cas 1 : \(s<\tau_1\).}
Ici \(\max(s,\tau_1)=\tau_1\) et \(\max(s,\tau_2)=\tau_2\). On a donc
\begin{align*}
T(s)&= B + C + A C - \alpha(B+C+AB+AC) - (1-\alpha)C(1+A+B)\\
&= \frac{AB(1 - A - B - C)}{B+2A}=0.
\end{align*}

\textbf{Cas 2 : \(\tau_1\le s < \tau_2\).}
Ici \(\max(s,\tau_1)=s\) et \(\max(s,\tau_2)=\tau_2\). On a
\begin{align*}
T(s)&= R(s) + AC - \alpha\big(1 + A\big)R(s) - (1-\alpha)\big(C + (A+B)C\big)\\
& = -\frac{A(1-A)C}{B+2A} + \frac{A(1-A-B)}{B+2A}R(s).
\end{align*}
Pour $R(s)=C$, on a
\(
T(s)= -\frac{A B C}{B+2A}\leq 0,
\)
et pour $R(s)=B+C$, on a
\(
T(s)=\frac{A B (1-A-B-C)}{B+2A}=0.
\)
Comme \(R(s) \in [C, B+C]\), on a $T(s)\leq 0$.

\bigskip
\textbf{Cas 3 : \(s \ge \tau_2\).}
Ici \(\max(s,\tau_1)=\max(s,\tau_2)=s\). On a
\begin{align*}
T(s)&=R(s)\Big[1 + A - \alpha\big(1 + A\big) - (1-\alpha)\big(1 + A+B\big)\Big]
\\
&= -\frac{AB}{B+2A}R(s) \leq 0.
\end{align*}
Ainsi, dans chacun des trois cas on a montré
\(T(s)\le 0\).
\end{proof}

Par les arguments précédents (symétrisation de la coalition et élimination des stratégies de type \(\tau_1<\tau_2\) par combinaison convexe), on peut se restreindre, sans perte de généralité, à l'étude du jeu réduit suivant :

\begin{itemize}
  \item la coalition est supposée jouer des profils \emph{symétriques} : tous ses \(K-1\) membres utilisent la même paire de seuils ;
  \item pour les deux camps, on ne considère que des stratégies pures \((\tau_1,\tau_2)\) telles que \(\tau_1\ge\tau_2\).
\end{itemize}

Rappelons que les seuils possibles prennent les valeurs entières
\(\{4,5,\dots,18\}\), soit \(15\) valeurs. Le nombre de couples \((\tau_1,\tau_2)\)
satisfaisant \(\tau_1\ge\tau_2\) est donc
\[
\#\{(\tau_1,\tau_2)\ :\ \tau_1,\tau_2\in\{4,\dots,18\},\ \tau_1\ge\tau_2\}
= \frac{15\cdot 16}{2}=120.
\]
\item
Par la question précédente, on est restreint à un jeu à 2 joueurs.
Pour $t,\tau_1\dots,\tau_k\in \{4,18\}^{\ell}$ on généralise la notion de dominance à $t\prec [\tau_1,\dots,\tau_k]$. On voit clairement que $t\prec [t,\tau_1,\dots,\tau_k] \Rightarrow t\prec [\tau_1,\dots,\tau_k].$ Par la question précédente, on a que si $t_i<t_{i+1}$ pour un $i\in \{1,\dots,\ell-1\},$ alors 
$(t_1,\dots,t_{\ell}) \prec [(t_1,\dots,t_{i-1},t_i,t_i,t_{i+2},\dots,t_{\ell}), (t_1,\dots,t_{i-1},t_{i+1},t_{i+1},t_{i+2},\dots,t_{\ell})].$ Plus généralement, on a le lemme suivant.
\begin{lemma}
Soit $a,b\in \{4,18\}$ avec $a<b$.
Soit $r\in \{2,\dots,\ell\}$ et $i\in \{0,\dots,\ell-r\}.$ Pour $x\in\{0,\dots,r\}$, on pose $$H(x)\triangleq (t_1,\dots,t_{i},\underbrace{a,\dots,a}_{x \text{ fois}},\underbrace{b,\dots,b}_{r-x \text{ fois}},t_{i+r+1},\dots,t_{\ell}).$$
Alors, $H(x)\prec [H(0),H(r)]$ pour $0<x<r.$
\end{lemma}
\begin{proof}
On veut montrer par récurrence sur $k\in \{2,\dots,r\}$ que 
\begin{align*}
&H(x)\prec [H(x'),H(x'')],~~ 0\leq x'<x<x''\leq r,~~ x''-x' = k.&(P_k)\end{align*}
Comme déjà mentionné, par la question précédente, on a $H(x)\prec [H(x-1),H(x+1)]$ pour $0<x<r,$ i.e., on a $(P_2)$. Si on a $(P_2),\dots,(P_k)$, alors soit $0\leq x'<x<x''\leq r,~~ x''-x' = k+1.$ 
\begin{itemize}
    \item Si $x'+1<x$, alors, par $(P_k)$, on a $H(x)\prec [H(x'+1),H(x'')]$. Par $(P_{x-x'}),$ on a $H(x'+1)\prec [H(x'),H(x)].$ On a donc $H(x) \prec[H(x'),H(x),H(x'')],$ et donc $H(x) \prec[H(x'),H(x'')].$ 
    \item  Si $x'+1=x$, alors, par $(P_2)$, on a $H(x)\prec [H(x'),H(x+1)].$
    Si $x+1=x''$, alors il n'y a rien à montrer.
Si $x+1<x''$, alors par $(P_{k})$, on a $H(x+1)\prec [H(x),H(x'')].$ On a donc $H(x)\prec [H(x'),H(x),H(x'')],$ et donc $H(x) \prec[H(x'),H(x'')].$ 
\end{itemize}
On a donc $(P_{k+1}).$ Par récurrence, on a donc $(P_r).$
\end{proof}
\begin{lemma}
Il suffit de considérer les stratégies pures de la forme $t=(t_1,\dots,t_{\ell})$ avec $t_1\geq \dots \geq t_{\ell}$. 
On peut donc se restreindre à un sous-ensemble de stratégies pures à $\binom{14 + \ell}{\ell}$ éléments.
\end{lemma}
\begin{proof}
On procède par récurrence sur le nombre d'indices $i$ tels que $t_i \neq t_{i+1}$. 
Si ce nombre vaut $0$,
il n'y a rien à montrer.
Supposons maintenant que ce nombre soit égal à $k+1$. Si les $k+1$ indices $i$ vérifient $t_i > t_{i+1}$, il n'y a rien à montrer. Sinon, un des indice $i$ vérifie $t_i < t_{i+1}$ et le lemme précédent permet alors de supprimer cet indice, ce qui conclut la récurrence.
\end{proof}
\item On utilise le code Python suivant.  
Sur un ordinateur portable classique, le calcul nécessite environ \(150\) secondes. Les équilibres obtenus sont résumés dans le tableau ci-dessous.

\begin{center}
\renewcommand{\arraystretch}{1.2}
\begin{tabular}{c|c|l}
\hline
$K$ & $t_1$ & Structure de $t_2(L)$ \\ 
\hline
2 & 13 & $13\{L=0\},~12\{L=1\}$ \\
3–5 & 12 & $12\{0\le L\le1\},~11\{2\le L\le K-1\}$ \\
6 & 12 & $12\{0\le L\le1\},~11\{2\le L\le4\},~10\{L=5\}$ \\
7–12 & 11 & $11\{0\le L\le4\},~10\{5\le L\le K-1\}$ \\
13–17 & 11 & $11\{0\le L\le4\},~10\{5\le L\le11\},~9\{12\le L\le K-1\}$ \\
18–30 & 10 & $10\{0\le L\le11\},~9\{12\le L\le K-1\}$ \\
31–48 & 10 & $10\{0\le L\le11\},~9\{12\le L\le29\},~8\{30\le L\le K-1\}$ \\
49–50 & 10 & $10\{0\le L\le11\},~9\{12\le L\le29\},~8\{30\le L\le49\}$ \\
51–87 & 9 & $9\{0\le L\le29\},~8\{30\le L\le K-1\}$ \\
88–100 & 9 & $9\{0\le L\le29\},~8\{30\le L\le87\},~7\{88\le L\le K-1\}$ \\
\hline
\end{tabular}
\end{center}

\begin{lstlisting}
(*@\codeheader{\currfiledir/4.py}@*)
(*@ @*)
\end{lstlisting}
\vspace{-.455cm}
\lstinputlisting{\currfiledir/4.py}

\end{enumerate}
\subsection*{Notes et références}
La situation étudiée est inspirée du deuxième wagon de la vidéo \textit{Stop the Train} (2025) de Squeezie, vidéaste français.  
D’un point de vue mathématique, cette énigme s’inscrit dans le cadre de la \emph{théorie des jeux}, branche des mathématiques qui modélise les interactions stratégiques entre agents rationnels~\cite{vonNeumannMorgenstern1944,Osborne2003}.  
Plus précisément, la question~\ref{Déterminer, en fonction de $K$} consiste à déterminer des \emph{équilibres de Nash}, notion introduite par John~Nash en 1950~\cite{Nash1950}.  
Un équilibre de Nash correspond à un profil de stratégies tel qu’aucun joueur ne puisse améliorer son gain en modifiant seul sa stratégie.  

Pour analyser la situation considérée à la question~\ref{On s’intéresse à la stratégie}, on met en œuvre la \emph{méthode d’élimination itérative des stratégies dominées}, procédé classique en théorie des jeux~\cite{osborne1994course}. 
Cette méthode consiste à retirer successivement les stratégies qui sont toujours moins performantes que d’autres, quels que soient les choix de l’adversaire.  
Dans le contexte de cette énigme, la notion de dominance entre stratégies est étroitement liée à celle de \emph{dominance stochastique}~\cite{QuirkSaposnik1962}, qui compare les lois de probabilité associées aux gains des joueurs.  
Cette relation permet de réduire significativement l’espace stratégique tout en conservant l’ensemble des équilibres pertinents du jeu.


\bibliography{\currfiledir/sources.bib}
\newpage




